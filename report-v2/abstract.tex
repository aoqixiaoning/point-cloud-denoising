\renewcommand{\abstracttextfont}{\normalfont}
\abstractintoc

% English
\begin{abstract}

% TODO
During this internship, we were interested in the denoising of point clouds and
more specifically an adaptive one: the magnitude of the smoothing will be more
important in some directions. The idea is to move the point cloud while
minimizing an energy. This energy will be related to the Minkowski sum of
the point set with a convex polyhedron.

This report is divided as follows: firstly, we will give a detailed introduction
on point cloud denoising, then we will focus on the state of the art related
techniques on smoothing and geometric flows.  Moreover, we will start by
studying the two dimensional case with the $r$-offset of a point cloud. We will
continue by dealing with the 3D case with a polyhedral norm.

\end{abstract}

\abstractintoc
\renewcommand\abstractname{R\'{e}sum\'{e}}

% Français
\begin{abstract} \selectlanguage{french}

% TODO
Durant ce stage, nous nous sommes intéressés au lissage adaptatif de nuage de points...

\end{abstract}

\selectlanguage{english}

% vim: set spelllang=en :
