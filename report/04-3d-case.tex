\chapter{3D case}

\section{Polyhedral norm}
A polyhedral norm is a norm $ N $ defined as the following:

$$ \forall x \in \mathbb{R}^d,~ N(x) = \max_{i} (x | v_i) $$

where the $ v_i $ are given vectors from $ \mathbb{R}^d $.

This definition implies that the unit ball for the norm $ N $ is a polyhedron.
Indeed, if $ x \in B_N(0, 1) $, then :

$$ N(x) \leq 1 \Longleftrightarrow \forall i, (x | v_i) \leq 1 $$

So, the unit ball is defined by linear constraints. Thus, it is an intersection
of halfspaces and so is a convex polyhedron. Hence, computing the $ r $-offset
of a point cloud under a polyhedral norm $ N $ is the same as computing the
Minkowski sum of a point cloud and a convex polyhedron.

\section{Construction of the Voronoi diagram for a polyhedral norm}

The main problem of a polyhedral norm is that the underlying Voronoi diagram is
not a partition of the space anymore.

For example, if we consider the $ L_\infty $ norm in the plane and two aligned
points, we have the following Voronoi region:
% TODO: insert figure

\subsection{Symbolic perturbation}

Given a polyhedral norm $ N $, we define its symbolic perturbation $ N_\epsilon
$ by: $ N_\epsilon (x) = N + \epsilon || x || $ where $ || \cdot || $ is an
euclidean norm (typically an $ L_p $ norm where $ 1 \leq p \leq +\infty $).

Then, we will show that the Voronoi cell of a point $ x $ under the norm $
N_\epsilon $ converges (when $ \epsilon \rightarrow 0 $) towards the Voronoi
cell defined as the following:

$$ V_{N'}(x, X) = \{ y \in \mathbb{R}^d,~\forall x' \in X,~N(x - y) \leq N(x'
-y) \text{ or } (N(x - y) = N(x' - y) \text { and } || x -y || \leq || x' - y ||) \} $$

First, we define the convergence of a set $ A $ towards another set $ B $: we
say that $ A $ converges towards $ B $ if the Hausdorff distance of $ A $ and $
B $ converges towards $ 0 $.

We recall that the Hausdorff distance is defined as follows:

$$ d_H(A, B) = \min \{ r \geq 0,~X \subset Y^{\oplus r} \text{ and } Y \subset
X^{\oplus r} \} $$

Now, we prove that $ V_{N_\epsilon}(x, X) $ converges towards $ V_{N'}(x, X) $
under the Hausdorff distance.

We want to prove that $ d_H(V_{N_\epsilon}(x, X), V_{N'}(x, X)) \rightarrow 0 $
Let's take a $ \delta > 0 $, we want to find $ \epsilon_\delta $ such that if $
\epsilon \leq \epsilon_\delta $ then $ d_H(V_{N_\epsilon}(x, X), V_{N'}(x, X))
\leq \delta $. The last inequality is equivalent to say that:
\begin{align}
    V_{N'}(x, X)) \subset V_{N_\epsilon}(x, X)^{\oplus \delta} \\
    V_{N_\epsilon}(x, X) \subset V_{N'}(x, X)^{\oplus \delta}
    \label{eqn:haussdorf-voronoi1}
\end{align}

Let's take care of the first inclusion:
% TODO: proof

And now the second inclusion:
Let's take $ y \in V_{N_\epsilon}(x, X) $. By definition of $ N_\epsilon $, we
have: $ \forall x' \in X,~N(x - y) + \epsilon || x - y || \leq N(x' - y) +
\epsilon || x' - y || $. If we choose $ z = y $ then $ z $ satisfies $ || z - y
|| \leq \delta $ and $ z \in V_{N'}(x, X) $. Indeed, using the previous
inequality, there are two cases: either $ N(x - y) \leq N(x' - y) $ or $ N(x -
y) = N(x' - y) $. In the latter case, the inequality reduces to: $ || x - y || \leq
|| x' - y || $. So $ y \in V_{N'}(x, X)^{\oplus \delta} $.

We also want to prove that this new Voronoi diagram has good properties, notably
this new diagram forms a good "partition" of $ \mathbb{R}^d $. Thus, we need to
define what a "good" partition is.
We can define it in two ways:
\begin{enumerate}
    \item for any point $ x $, $ V_{N'}(x, X) $ is homeomorphic to a disk. For
        any couple $ (x, y) $, $ V_{N'}(x, X) \cap V_{N'}(y, X) $ is
        homeomorphic to a segment / half-line / line. For any triplet $ (x, y,
        z) $, $ V_{N'}(x, X) \cap V_{N'}(y, X) \cap V_{N'}(z, X) $ is
        homeomorphic to a point.
    \item Let's consider the open Voronoi regions: $ \ocirc{V}_N(x, X) = \{ y
        \in X,~ N(x - y) < N(x' - y)\} $.
\end{enumerate}
% TODO: definition

\subsection{Construction of a Voronoi cell}

The goal of this part will be to provide an incremental algorithm for
constructing the Voronoi cell $ V_{N'}(x, X)) $ in 3D using a 2D triangulation
on a sphere.

If we project the Voronoi cell onto the unit sphere, we will obtain a mesh.

% TODO: normal cone

We will need some definitions:
\begin{itemize}
    \item normal cone to a convex set $ K $
\end{itemize}

\section{Experiments}

% Construction of the Voronoi diagram
% -> polyhedral norm
% -> construction
% -> r-offset for the norm N
% -> volume
% -> evolution
% Experiments

% vim: set spelllang=en :
