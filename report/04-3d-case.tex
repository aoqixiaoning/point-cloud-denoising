\chapter{3D case}

In this part, we will be interested in doing the same job in 3D as in 2D except
that we will use a convex polyhedron instead of an euclidean ball in the
Minkowski sum.

This choice will raise a number of issues such as the construction of the
Voronoi diagram of a point set under a polyhedral norm.

\section{Polyhedral norm}
A polyhedral norm is a norm $ N $ defined as the following:

$$ \forall x \in \mathbb{R}^d,~ N(x) = \max_{i} (x | v_i) $$

where the $ v_i $ are given vectors from $ \mathbb{R}^d $.

This definition implies that the unit ball for the norm $ N $ is a polyhedron.
Indeed, if $ x \in B_N(0, 1) $, then :

$$ N(x) \leq 1 \Longleftrightarrow \forall i, (x | v_i) \leq 1 $$

So, the unit ball is defined by linear constraints. Thus, it is an intersection
of halfspaces and so is a convex polyhedron. Hence, computing the $ r $-offset
of a point cloud under a polyhedral norm $ N $ is the same as computing the
Minkowski sum of a point cloud and a convex polyhedron.

\section{Construction of the Voronoi diagram for a polyhedral norm}

The main problem of a polyhedral norm is that the underlying Voronoi diagram is
not a partition of the space anymore.

For example, if we consider the $ L_\infty $ norm in the plane and two aligned
points, we have the following Voronoi diagram:

% TODO: insert figure

\subsection{Symbolic perturbation}

Given a polyhedral norm $ N $, we define its symbolic perturbation $ N_\epsilon
$ by: $ N_\epsilon (x) = N + \epsilon || x || $ where $ || \cdot || $ is an
euclidean norm (typically an $ L_p $ norm where $ 1 \leq p \leq +\infty $).

Then, we will show that the Voronoi cell of a point $ x $ under the norm $
N_\epsilon $ converges (when $ \epsilon \rightarrow 0 $) towards the Voronoi
cell defined as the following:

$$ V_{N'}(x, X) = \{ y \in \mathbb{R}^d,~\forall x' \in X,~N(x - y) \leq N(x'
-y) \text{ or } (N(x - y) = N(x' - y) \text { and } || x -y || \leq || x' - y ||) \} $$

First, we define the convergence of a set $ A $ towards another set $ B $: we
say that $ A $ converges towards $ B $ if the Hausdorff distance of $ A $ and $
B $ converges towards $ 0 $.

We recall that the Hausdorff distance is defined as follows:

$$ d_H(A, B) = \min \{ r \geq 0,~X \subset Y^{\oplus r} \text{ and } Y \subset
X^{\oplus r} \} $$

Now, we prove that $ V_{N_\epsilon}(x, X) $ converges towards $ V_{N'}(x, X) $
under the Hausdorff distance.

We want to prove that $ d_H(V_{N_\epsilon}(x, X), V_{N'}(x, X)) \rightarrow 0 $
Let's take a $ \delta > 0 $, we want to find $ \epsilon_\delta $ such that if $
\epsilon \leq \epsilon_\delta $ then $ d_H(V_{N_\epsilon}(x, X), V_{N'}(x, X))
\leq \delta $. The last inequality is equivalent to say that:
\begin{align}
    V_{N'}(x, X)) \subset V_{N_\epsilon}(x, X)^{\oplus \delta} \\
    V_{N_\epsilon}(x, X) \subset V_{N'}(x, X)^{\oplus \delta}
    \label{eqn:haussdorf-voronoi1}
\end{align}

Let's take care of the first inclusion:
% TODO: proof

And now the second inclusion:
Let's take $ y \in V_{N_\epsilon}(x, X) $. By definition of $ N_\epsilon $, we
have: $ \forall x' \in X,~N(x - y) + \epsilon || x - y || \leq N(x' - y) +
\epsilon || x' - y || $. If we choose $ z = y $ then $ z $ satisfies $ || z - y
|| \leq \delta $ and $ z \in V_{N'}(x, X) $. Indeed, using the previous
inequality, there are two cases: either $ N(x - y) \leq N(x' - y) $ or $ N(x -
y) = N(x' - y) $. In the latter case, the inequality reduces to: $ || x - y || \leq
|| x' - y || $. So $ y \in V_{N'}(x, X)^{\oplus \delta} $.

We also want to prove that this new Voronoi diagram has good properties, notably
this new diagram forms a good "partition" of $ \mathbb{R}^d $. Thus, we need to
define what a "good" partition is.
We can define it in two ways:
\begin{enumerate}
    \item for any point $ x $, $ V_{N'}(x, X) $ is homeomorphic to a disk. For
        any couple $ (x, y) $, $ V_{N'}(x, X) \cap V_{N'}(y, X) $ is
        homeomorphic to a segment / half-line / line. For any triplet $ (x, y,
        z) $, $ V_{N'}(x, X) \cap V_{N'}(y, X) \cap V_{N'}(z, X) $ is
        homeomorphic to a point.
    \item Let's consider the open Voronoi regions: $ \ocirc{V}_N(x, X) = \{ y
        \in X,~ N(x - y) < N(x' - y)\} $. Then, we can define a good partition
        by saying that the complement of the open Voronoi regions must have a
        zero measure.
\end{enumerate}
% TODO: definition

\subsection{Construction of a Voronoi cell}

The construction of the Voronoi diagram under a polyhedral norm is a complex
topic and had been studied thoroughly in \cite{ma2000bisectors}.

For the duration of this internship, we had the choice between three paths
(by increasing order of difficulty):
\begin{enumerate}
    \item Construction of $ V(p, P) \cap B(p, r) $
    \item Construction of $ V(p, P) \cap B_N(p, r) $
    \item Construction of $ V_N(p, P) \cap B_N(p, r) $
\end{enumerate}

We chose the second path because the third one, as said previously, is a very
complex subject and can not be done during the internship because of its length.
Also, we can approximate the first one using the second one by choosing for $
B_N(p, r) $ a discretization of a sphere.

We used two different methods in order to evaluate the volume (or the area of
the boundary or more generally an additive measure $ \mu $) of $ V(p, P) \cap
B_N(p, r) $: a simple method by directly summing the quantities $ \mu(V(p, P)
\cap B_N(p, r)) $ or by using a variant of the inclusion-exclusion formula.

\subsubsection{Naive method}
% TODO

We want to compute :

$$ \mu(P + r B_N(0, 1)) = \sum_p \mu(\bigcup_p B_N(p, r) \cap V(p, P)) $$

The idea of this method is to approximate $ \mu(\bigcup_p B_N(p, r) \cap V(p,
P)) $ by $ \mu(B_N(p, r) \cap V(p, P)) $.

\subsubsection{Inclusion-exclusion formula}
% TODO

The inclusion-exclusion formula is a well known formula which describes the
indicator function of a union of sets: given a finite number of sets $ A = \{
A_1, \ldots, A_N \} $, we have:

$$ \indicator{\bigcup A_i} = \sum_{\emptyset \neq X \subseteq A} (-1)^{card X -
    1} \indicator{\bigcap X} $$

This formula can also be expressed using the notion of nerve as shown in
\cite{attali2007inclusion}. We defined the nerve of $ A = \{ A_x, x \in X \} $
to be the graph where an edge exists between $ x $ and $ y $ only if $ A_x \cap
A_y \neq \emptyset $.

Then, we can write the inclusion-exclusion formula as:

$$ \indicator{\bigcup A_x} = \sum_{\sigma \in Nerve(A)} (-1)^{\dim \sigma}
\indicator{\bigcap \sigma} $$

In our case, we get:

\begin{equation}
    \indicator{\bigcup B_N(p, r)} = \sum_{\sigma \in Nerve(B)} (-1)^{\dim \sigma}
    \indicator{\bigcap \sigma}
    \label{eqn:incl_excl_simplices}
\end{equation}

where $ B $ is the collection of all the $ B_N(p, r), p \in P $.

Now, the approximation is to replace $ Nerve(B) $ by $ Del(P, r) $ which is the
$\alpha$-complex of $ P $ with $ \alpha = r $.

% TODO: explain why approximation

A question we can ask is: is the inclusion-exclusion formula still valid if we
replace $ Nerve(B) $ with $ Nerve(\{ B_N(p, r) \cap V(p, P), p \in P\}) $ ?

If we want to prove this assertion, we may use the technique used to prove the
inclusion-exclusion formula for union of balls.

The proof starts by defining the subcomplex $ L_p $ induced by a vertex $ p $.
We consider all the polyhedrons that contains $ p $ and we construct the nerve of
it. Formally, $ L_p = Nerve(\{ B_N(x, r), p \in B_n(x, r)\}) $.

Then, when we evaluate the indicator function at $ p $ of the union, we can
decompose it in two parts: simplices of $ L_p $ and the other ones.

\begin{align*}
    \indicator{\bigcup B_N(p, r)}(p) &= \sum_{\sigma \in L_p} (-1)^{\dim \sigma}
    \indicator{\bigcap \sigma}(p) + \sum_{\sigma \notin L_p} (-1)^{\dim \sigma}
    \underbrace{\indicator{\bigcap \sigma}(p)}_{= 0 \text{ since } p \notin
        \sigma} \\
    &= \sum_{\sigma \in L_p} (-1)^{\dim \sigma} \\
    &= \chi(L_p)
\end{align*}

Here, $ \chi $ is the Euler characteristic of $ L_p $.

Now, if we show that $ L_p $ is contractible (can be continuously deformed to a
point), we will get $ \chi(L_p) = 1 $. Obviously, we also have $
\indicator{\bigcup B_N(p, r)}(p) = 1 $ and we conclude that the formula
\ref{eqn:incl_excl_simplices} is true.

We can use this argument to generalize this formula. If we suppose the there
exists $ r, r' $ such that: $ B(p, r) \subseteq B_N(p, r) \subseteq B(p, r') $,
then we have: $ K_p \subseteq L_p \subseteq K'_p $ where $ K_p $ (resp. $ K'_p $)
is the induced subcomplex of $ p $ in the simplicial complex defined by all the
balls $ B(p, r) $ (resp. $ B(p, r') $), $ p \in P $.

Then, we know that $ K_p $ and $ K'_p $ are either empty or contractible. We
want to deduce that $ L_p $ is either empty or contractible.

% TODO: finish the proof

\section{Experiments}

% TODO

We compare the two different methods.

% Construction of the Voronoi diagram
% -> polyhedral norm
% -> construction
% -> r-offset for the norm N
% -> volume
% -> evolution
% Experiments

% vim: set spelllang=en :
