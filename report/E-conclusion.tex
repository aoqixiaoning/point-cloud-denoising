\chapter{Conclusion \& Perspectives}

During this internship, we developed different algorithms for point cloud
smoothing. In Chapter \ref{chapter:2d}, we looked at one which allows us to
estimate the mean curvature and smooth 2D point clouds using a discretization of
the mean curvature flow. Instead of minimizing a continuous area, we minimize a
discrete one: the volume of a union of balls. In order to do that, we use a
gradient descent based algorithm combined with the Automatic Differentiation
technique.

Then, in Chapter \ref{chapter:3d}, we extended this algorithm in 3D by replacing
the union of balls with a union of convex polyhedra. The obtained flow has
different properties than the normal mean curvature flow: it is anisotropic. We
showed that the choice of the convex polyhedron influence the directions of the
anisotropic smoothing.

Finally, in Chapter \ref{chapter:theory}, we examined the theoretical properties
of our discretization of the area. We proved that the approximation of the area
of a surface by the volume of a union of balls is a good under some conditions.
We also proved that the flow we constructed in Chapter \ref{chapter:2d} is
indeed a discretization of the mean curvature flow: the gradient of the volume
of a union of balls converge towards a quantity proportional to the mean
curvature vector.

There is still some place left for improvements:
\begin{itemize}
    \item We did the proof of the convergence for the gradient of the volume of
        the union (see Proposition \ref{prop:gradient-mean-curvature}). It would
        be interesting to study the convergence of the gradient of the area of
        the boundary. Our experiments seem to show that this gradient tends to
        converge also to a quantity proportional to the mean curvature vector.
    \item Correct the algorithm in 3D in order to minimize the good functional.
        Indeed, we saw in Section \ref{sec:theory-3d-case} that we do not
        minimize the volume of the union of convex polyhedra but another,
        related functional. It would be interesting to manage to find a way to
        use the true functional.
    \item Implement an exact computation of the volume of the union in 3D using
        arrangements and overlays.
    \item Use the 3D algorithm to simulate crystal growth in physics.
\end{itemize}

% vim: set spelllang=en :
