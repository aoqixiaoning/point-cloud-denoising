\chapter{Automatic Differentiation}
\label{appendix:ad}

The automatic differentiation is a technique used for computing the derivatives,
gradient of programs. It relies on the fact that no matter how complex a program
is, it is composed of a sequence of elementary arithmetic operations (such as
addition, subtraction, multiplication, division) and elementary functions (exp,
log, cos, sin...) It extensively uses the chain rule on these operations to
compute derivatives of arbitrary order.

It is not the same as the symbolic or the numerical differentiation (like finite
differences). Indeed, automatic differentiation can be used to compute
derivatives of programs and not only mathematical functions without using
approximations like with numerical differentiation.

Let us suppose that we want to compute the derivative of a function $ f $ with a
single one dimensional argument $ x $. To do that, we will replace the number
type that is to say that we will replace $ x $ by $ x + \epsilon x' $ where $
\epsilon $ satisfies the property $ \epsilon^2 = 0 $ ($ x + \epsilon x' $ is
said to be a dual number). Then, we overload all the arithmetical operations:
addition, subtraction, multiplication, division.

Then we call $ f $ with this variable: $ f(x + \epsilon x') = y + \epsilon y' $.
We conclude that $ y = f(x) $ and $ y' = x' f'(x) $ . Indeed for small values of
$ \epsilon $, we have the following first order Taylor expansion: $ f(x +
\epsilon x') = f(x) + \epsilon x' f'(x) + ... $. $ x' $ is called a seed and can
be chosen arbitrarily, we can for instance choose $ x ' = 1 $ and so $ y' =
f'(x) $.

This process can be easily extended to handle functions like $ f : \R^n
\rightarrow \R $ in order to compute gradients of such functions. This is what
we did during the internship: we considered a function over $ n $ points in
dimension $ d $ as a function $ f : \R^{dn} \rightarrow \R $.

This technique is interesting because it allows us to compute accurate
derivatives of functions very easily and efficiently. Indeed, we only have to
change the number type and write the good implementations of the basic
arithmetic operations and overloaded mathematical functions.  Another advantage
is that in \texttt{CGAL}, this is easy to do since the library can be
parametrized by the number type.

% TODO

% vim: set spelllang=en filetype=tex :
