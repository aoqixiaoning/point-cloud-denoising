\renewcommand{\abstracttextfont}{\normalfont}
\abstractintoc

% English
\begin{abstract}

During this internship, we were interested in the denoising of point clouds and
more specifically an adaptive one: the smoothing is more important at a higher
scale. The idea is to move the point cloud in while minimizing an energy. The
energy we considered here was given by the volume of the Minkowski sum of the
point set with a convex polyhedron.

In order to efficiently construct this sum, we had to construct the Voronoi
diagram for a polyhedral norm.

This report is divided as follows: firstly, we will give a detailed
introduction, then we will focus on the state of the art related techniques.
Moreover, we will start by studying the two dimensional case with the $r$-offset
of a point cloud. We will continue by dealing with the 3D case with a polyhedral
norm.

\end{abstract}

\abstractintoc
\renewcommand\abstractname{R\'{e}sum\'{e}}

% Français
\begin{abstract} \selectlanguage{french}
% TODO
Durant ce stage, nous nous sommes intéressés au lissage de nuage de points.
\end{abstract}

\selectlanguage{english}

% vim: set spelllang=en :
