\chapter{Voronoi diagrams for polyhedral norms}
\label{appendix:voronoi-polyhedral-norm}

Given a polyhedral norm $ N $, we define its symbolic perturbation $ N_\epsilon
$ by: $ N_\epsilon (x) = N + \epsilon || x || $ where $ \epsilon > 0 $ and  $ ||
\cdot || $ is an Euclidean norm (typically an $ L_p $ norm where $ 1 < p <
+\infty $).  We will show that the Voronoi cell of a point $ x $ under the norm
$ N_\epsilon $ converges (when $ \epsilon \rightarrow 0 $) towards the Voronoi
cell defined as the following:

$$ V_{N'}(x, X) = \{ y \in \mathbb{R}^d,~\forall x' \in X,~N(x - y) \leq N(x'
-y) \text{ or } (N(x - y) = N(x' - y) \text { and } || x -y || \leq || x' - y ||) \} $$

First, we define the convergence of a set $ A $ towards another set $ B $: we
say that $ A $ converges towards $ B $ if the Hausdorff distance of $ A $ and $
B $ converges towards $ 0 $. We recall that the Hausdorff distance is defined as
follows:

$$ d_H(A, B) = \min \{ r \geq 0,~X \subset Y^r \text{ and } Y \subset X^r \} $$

Now, we prove that $ V_{N_\epsilon}(x, X) $ converges towards $ V_{N'}(x, X) $
under the Hausdorff distance. We want to prove that $ d_H(V_{N_\epsilon}(x, X),
V_{N'}(x, X)) \to 0 $.  Let us take a $ \delta > 0 $, we want to find $
\epsilon_\delta $ such that if $ \epsilon \leq \epsilon_\delta $ then $
d_H(V_{N_\epsilon}(x, X), V_{N'}(x, X)) \leq \delta $. The last inequality is
equivalent to:
\begin{align}
    V_{N'}(x, X)) \subset V_{N_\epsilon}(x, X)^\delta \\
    V_{N_\epsilon}(x, X) \subset V_{N'}(x, X)^\delta
    \label{eqn:haussdorf-voronoi1}
\end{align}

Let us take care of the first inclusion. Let us take $ y \in V_{N'}(x, X) $. By
definition we have, for a given $ x' \in X $:
$$
N(x - y) \leq N(x' -y) \text{ or } (N(x - y) = N(x' - y) \text { and } || x -y
|| \leq || x' - y ||)
$$
So, we have two cases:
\begin{itemize}
    \item $ N(x - y) \leq N(x' - y) $: we did not have the time to look at this
        case during this internship. But we think that we can show that for
        sufficiently small values of $ \epsilon $, we have: $ N(x - y) +
        \epsilon || x  - y || \leq N(x' - y) \leq || x ' - y || $, then $ y \in
        V_{N_\epsilon}(x, X) $.
    \item $ N(x - y) = N(x' - y) $ and $ || x - y || \leq || x' - y || $: in
        this case, we have:
        \begin{align*}
            N_\epsilon(x - y) &= N(x - y) + \epsilon || x - y || \\
            &= N(x' - y) + \epsilon|| x - y || \\
            & \leq N(x' - y) + \epsilon || x' - y || = N_\epsilon(x' - y) \\
            & \implies y \in V_{N_\epsilon}(x, X) \\
            & \implies y \in V_{N_\epsilon}(x, X)^\delta \text{ for any } \delta
            \geq 0
        \end{align*}
\end{itemize}

And now the second inclusion. Let us take $ y \in V_{N_\epsilon}(x, X) $. By
definition of $ N_\epsilon $, we have: $ \forall x' \in X,~N(x - y) + \epsilon
|| x - y || \leq N(x' - y) + \epsilon || x' - y || $. Then, we will show that $
y \in V_{N'}(x, X) $. Indeed, using the previous inequality, there are two
cases: either $ N(x - y) \leq N(x' - y) $ or $ N(x - y) = N(x' - y) $. In the
former case, we have directly that $ y \in V_{N'}(x, X) $ by definition. In the
latter case, the inequality reduces to: $ || x - y || \leq || x' - y || $ and so
$ y \in V_{N'}(x, X) $ i.e. $ y \in V_{N'}(x, X)^{\delta} $ for any $ \delta $.

We can also prove that this new Voronoi diagram has good properties, notably
that this new diagram forms a "partition" of $ \mathbb{R}^d $. "Partition",
here, must be understand in the following sense: for any $ p, q \in P $, $ p
\neq q $, the intersection $ V_{N'}(p, P) \cap V_{N'}(q, P) $ has a zero measure.
This property allows us to use the formula:

$$ Vol(P+rB_N(0, 1)) = \sum_{p \in P} Vol(V_{N'}(p, P) \cap B_N(p, r)) $$

Indeed, if the Voronoi diagram was not a partition, then some regions would have
been counted twice.

% vim: set spelllang=en filetype=tex :

