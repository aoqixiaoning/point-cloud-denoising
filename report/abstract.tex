\renewcommand{\abstracttextfont}{\normalfont}
\abstractintoc

% English
\begin{abstract}

During this internship, we were interested in the denoising, smoothing of point
clouds sampled on an unknown surface. In order to do that, we will use a mean
curvature flow approach: points are moved while minimizing an energy, this
energy being the area of the underlying surface. But, we do not know the
underlying surface, only points sampled on it. To approximate the area, we will
use the volume of a union of balls centered on the point cloud and a gradient
descent algorithm.  We will also be interested in another kind of flow: an
anisotropic one. For this particular flow, the idea is to replace the union of
balls by a union of convex polyhedra. The magnitude of the smoothing will be
more important in some directions dictated by the choice of a convex polyhedron.

This report is divided as follows. Firstly, we will give a detailed introduction
about point cloud smoothing: the existing techniques and why using a mean
curvature flow can be interesting. Then, we will study the two dimensional case.
We will show that this technique can be used to simulate a discrete mean
curvature flow and to smooth point clouds. We will also see that this technique
can be used to estimate the mean curvature. Many examples will be studied.
Secondly, we will look at the anisotropic flow in  3D. Several theoretical and
practical issues will arise. We will explain the choices that were mode to solve
these issues.  Finally, we will look at the theory behind our approach. We will
prove that, under some conditions, the area of a surface can be well
approximated by the volume of a union of balls. We will also prove that our
algorithm is, indeed, a discretization of the continuous mean curvature flow.

\end{abstract}

\newpage
\abstractintoc
\renewcommand\abstractname{R\'{e}sum\'{e}}

% Français
\begin{abstract} \selectlanguage{french}

Durant ce stage, nous nous sommes intéressés au débruitage, lissage de nuages de
points échantillonnés sur une surface inconnue. Pour faire cela, nous avons
utilisé une approche basée sur le flot de courbure moyenne: nous faisons évoluer
le nuage de points tout en minimisant une énergie, l'aire de la surface
sous-jacente. Mais, dans notre cas, la surface est inconnue, nous ne connaissons
que des points échantillonnés sur celle-ci. Pour approcher l'aire de la surface,
nous avons utilisé le volume de l'union des boules centrées sur le nuage de
point et un algorithme basé sur une descente de gradient. Nous nous sommes aussi
intéressé à un autre type de flot: un flot anisotropique.  Pour ce flot, l'idée
est de remplacer l'union des boules par une union de polyèdres convexes. La
magnitude du flot sera plus importante dans certaines directions privilégiées,
dictées par le choix du polyèdre.

Ce rapport est structuré de la manière suivante: tout d'abord, nous donnerons
une introduction détaillée sur les techniques existantes de lissage de nuages de
points et pourquoi il est intéressant d'avoir choisi une approche basée sur le
flot de courbure moyenne. Ensuite, nous étudierons le cas bidimensionnelle. Nous
montrerons que cette technique peut être utilisée pour simuler un flot de
courbure moyenne discret et de lisser des nuages de points. Nous verrons
aussi que cette technique peut être utilisée pour estimer la courbure
moyenne. Beaucoup d'exemples illustreront ces résultats. Ensuite, nous nous
intéresserons au cas anisotropique en 3D. Nous serons confrontés à des
difficultés aussi bien théoriques que pratiques. Nous expliquerons alors les
choix qui ont été faits pour résoudre les problèmes rencontrés. Enfin, nous nous
intéresserons à la théorie utilisée derrière notre approche. Nous montrerons
que, sous certaines conditions, l'aire de la surface peut être approchée par le
volume d'une union de boules. Nous prouverons aussi que notre algorithme est
bien une discrétisation du flot de courbure moyenne continu.

\end{abstract}

\selectlanguage{english}

% vim: set spelllang=en :
