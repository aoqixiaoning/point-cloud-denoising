\chapter{Automatic Differentiation}
\label{appendix:ad}

The automatic differentiation is a technique used for computing the derivatives
functions. It relies on the fact that no matter how complex a computer program
is, it is composed of a sequence of elementary arithmetic operations (such as
addition, subtraction, multiplication, division) and elementary functions (exp,
log, cos, sin...). Automatic Differentiation extensively uses the chain rule on
these operations to compute derivatives.

It is not the same as symbolic or numerical differentiation (like finite
differences). Indeed, automatic differentiation can be used to compute
derivatives of programs and not only mathematical functions without using
approximations as in numerical differentiation.

Let us suppose that we want to compute the derivative of a function $ f $ with a
single one dimensional argument $ x $. To do that, we will replace the number
type that is to say that we will replace $ x $ by $ x + \epsilon x' $ where $
\epsilon $ satisfies the property $ \epsilon^2 = 0 $ ($ x + \epsilon x' $ is
said to be a dual number). Then, we overload all the arithmetical operations:
addition, subtraction, multiplication, division. For instance: $ (x + \epsilon
x') \times (y + \epsilon y') = x y + \epsilon (x y' + x' y) $.

Then we call $ f $ with this number, the result is also a dual number: $ f(x +
\epsilon x') = y + \epsilon y' $.  Furthermore, for small values of $ \epsilon
$, we have the following first order Taylor expansion: $ f(x + \epsilon x') =
f(x) + \epsilon x' f'(x) + ... $ By identification, we get $ y = f(x) $ and $
y' = x' f(x) $. $ x' $ is called a seed and can be chosen arbitrarily, we can
for instance choose $ x ' = 1 $ so that $ y' = f'(x) $.

This process can be easily extended to handle functions like $ f : \R^n
\rightarrow \R $ in order to compute gradients. This is what we used during the
internship: we considered a function over $ n $ points in dimension $ d $ as a
function $ f : \R^{dn} \rightarrow \R $.

This technique is interesting because it allows us to compute accurate
derivatives of functions very easily and efficiently. Indeed, we only have to
change the number type and write the good implementations of the basic
arithmetic operations and overloaded mathematical functions. It also allows us
to only focus on the computation of the value and not its derivative. Another
advantage is that in \texttt{CGAL}, this is easy to do since the library can be
parametrized by the number type.

% vim: set spelllang=en filetype=tex :
