\chapter{State of the art}

% TODO: references

Nowadays, the traditional methods used for smoothing a point cloud are the
following:
\begin{itemize}
    \item Moving Least Squares (MLS) and its adaptive variant
    \item Laplacian smoothing
\end{itemize}

In this internship, we chose to use a Mean Curvature Flow (MCF) based technique.

The MCF is well known for its smoothing properties used for example in image
processing (see \cite{ciomaga2010level}).

We can also relate the MCF to the gradient flow of the volume area functional.

Let's recall the more important results related to the MCF.  First the
definition of this flow: given a surface. Each step of the flow will consist in
moving each point of the surface in the direction of its inward normal by an
amount related to the mean curvature at this point.

Multiple approaches exist:
\begin{itemize}
    \item Level-set approach: the moving surface is represented by : $ \{ x |
        \phi(x, t) = 0 \} $. Then $ \phi $ satisfies the following equation: $
        \frac{\partial \phi}{\partial t} = |\nabla \phi| div(\frac{\nabla
            \phi}{| \nabla \phi |}) $.
    \item Graph approach: the surface is a graph of a function $ f : U
        \rightarrow \R $. Then $ f $ satisfies the following equation: $
        \frac{\partial f}{\partial t} = \sqrt{|\nabla f|^2 + 1} ~div(\frac{\nabla
        f}{\sqrt{|\nabla f|^2 + 1}}) $.
\end{itemize}

These approaches are mainly based on solving differential equations and
continuous ones. In this report, we will focus on computational geometry related
techniques on "discrete" surfaces (i.e. point clouds).

% vim: set spelllang=en :
